\documentclass{article}

% Language setting
% Replace `english' with e.g. `spanish' to change the document language
\usepackage[english]{babel}

% Set page size and margins
% Replace `letterpaper' with `a4paper' for UK/EU standard size
\usepackage[letterpaper,top=2cm,bottom=2cm,left=3cm,right=3cm,marginparwidth=1.75cm]{geometry}

% Useful packages
\usepackage{amsmath}
\usepackage{graphicx}
\usepackage[colorlinks=true, allcolors=blue]{hyperref}

\title{Google TPU}
\author{Fernando Lima \\ Isabella Caselli \\ Rodrigo Michelassi}

\begin{document}
\maketitle

\begin{abstract}
Na era do desenvolvimento de sistemas baseados em Inteligência Artificial, se faz necessário o uso de máquinas super potentes, capazes de processar dados e realizar operações matemáticas de maneira extremamente rápida. Modelos de Machine Learning podem levar horas, até mesmo dias, para serem treinados, devido principalmente a operações como produto interno entre matrizes e a enorme quantidade de dados que são usados, trazendo um prejuízo não apenas de tempo, mas também energético, ambiental e sobretudo lucrativo. Nesse artigo, iremos tratar brevemente sobre a utilização de Cloud TPUs, unidades de processamento de tensores do Google Cloud, que atuam na otimização do treinamento de modelos de aprendizado de máquina, e que se tornou indispensável na academia e na indústria, para todos estudiosos e profissionais da área.
\end{abstract}

\section{Introdução}

Your introduction goes here! Simply start writing your document and use the Recompile button to view the updated PDF preview. Examples of commonly used commands and features are listed below, to help you get started.

Once you're familiar with the editor, you can find various project settings in the Overleaf menu, accessed via the button in the very top left of the editor. To view tutorials, user guides, and further documentation, please visit our \href{https://www.overleaf.com/learn}{help library}, or head to our plans page to \href{https://www.overleaf.com/user/subscription/plans}{choose your plan}.

%Chapter{History}

\section{Tensores}

\section{TPU vs GPU}

\section{TensorFlow}

\section{Cloud TPU v5p}

\section{Google Colab e distribuição}

\subsection{How to add Citations and a References List}

You can simply upload a \verb|.bib| file containing your BibTeX entries, created with a tool such as JabRef. You can then cite entries from it, like this: \cite{greenwade93}. Just remember to specify a bibliography style, as well as the filename of the \verb|.bib|. You can find a \href{https://www.overleaf.com/help/97-how-to-include-a-bibliography-using-bibtex}{video tutorial here} to learn more about BibTeX.

If you have an \href{https://www.overleaf.com/user/subscription/plans}{upgraded account}, you can also import your Mendeley or Zotero library directly as a \verb|.bib| file, via the upload menu in the file-tree.

\subsection{Good luck!}

We hope you find Overleaf useful, and do take a look at our \href{https://www.overleaf.com/learn}{help library} for more tutorials and user guides! Please also let us know if you have any feedback using the Contact Us link at the bottom of the Overleaf menu --- or use the contact form at \url{https://www.overleaf.com/contact}.

\bibliographystyle{alpha}
\bibliography{sample}

\end{document}